\documentclass{article}
\usepackage[utf8]{inputenc}
\usepackage[siunitx]{circuitikz}
\usepackage{lmodern}

\usepackage{amsmath}
\usepackage{soul}
\usepackage{amssymb}

\usepackage{mathrsfs}
\usepackage{siunitx}

\usepackage{lettrine}
\usepackage{graphicx}%permet de mettre des images, les réduire, les placer, les tourner [scale=....,angle=...]
\usepackage{pdfpages}%permet d'inclure des pages de documents pdfs
\usepackage{booktabs}%permet de faire des jolis tableaux
\usepackage{multicol}%permet de faire des listes sur 2 colonnes
\usepackage{multirow}
\usepackage{geometry}%changer les marges, footers, headers
\usepackage{float} % permet de choisir mieux ou mettre les images avec[H],t,b,p,
\usepackage[T1]{fontenc}
\usepackage{enumitem}
\usepackage{xcolor}%mettre du txt en couleur \textcolor{colorname}{Text to be colored}
\usepackage{textcomp,gensymb} %permet le \degree \perthousand \micro
\usepackage{amsmath}%faire des équas de math alignées et des théorèmes
\usepackage[french]{babel}%met le document en francais (table de matières
\usepackage[T1]{fontenc}%pour avoir les accents et les caractères bizarres
\usepackage{rotating}% permet de tourner l'image avec \begin{sidewaysfigure}
\usepackage{fancyhdr}%modifier les headers et footers
\usepackage{hyperref}%permet de cliquer sur les urls et les sections dans les pdfs
\usepackage{titling}
\usepackage{wrapfig}
\usepackage{listingsutf8}  % Insertion de code
\lstset{
	inputencoding=utf8/latin1,
	showstringspaces=false,
	basicstyle=\ttfamily,
	keywordstyle=\color{blue},
	commentstyle=\color[grey]{0.6},
	stringstyle=\color[RGB]{255,150,75}
} % Setup de l'encodage des fragments de code
\usepackage{nameref}

\setlength{\headheight}{12.2pt}
\setlength{\droptitle}{-1.5cm}
\geometry{margin=2cm}


\begin{document}
% Comme ce fichier est géré avec Git, il est un bon usage de faire 1 phrase = 1 ligne au lieu d'écrire les paragraphes sur une ligne comme on peut le faire habituellement).
% Cela permet à 'git diff' de fonctionner correctement.


% Template issu de: https://www.latextemplates.com/template/academic-title-page
\begin{titlepage} % Suppresses displaying the page number on the title page and the subsequent page counts as page 1
	\newcommand{\HRule}{\rule{\linewidth}{0.5mm}} % Defines a new command for horizontal lines, change thickness here
	
	%------------------------------------------------
	%   Logos
	%------------------------------------------------
	\begin{minipage}[t]{0.30\linewidth}
		\includegraphics[height=1.5cm]{Images/logo_EPL.jpg}
	\end{minipage} \hfill
	\begin{minipage}[t]{0.35\linewidth}
		\includegraphics[height=1.5cm]{Images/logo_UCL.jpg}
	\end{minipage}
	
	\center % Centre everything on the page
	
	
	%------------------------------------------------
	%	Headings
	%------------------------------------------------
	
	%\textsc{\LARGE Ecole Polytechnique de Louvain}\\[1.5cm] % Main heading such as the name of your university/college
	
	\vspace{1.5cm}
	\textsc{\Large Systèmes Informatiques I}\\[0.5cm] % Major heading such as course name
	
	\textsc{\large LSINF1252}\\[1.0cm] % Minor heading such as course title
	
	%------------------------------------------------
	%	Title
	%------------------------------------------------
	
	\HRule\\[0.65cm]
	%\vspace{1.5cm}
	
	{\huge\bfseries Password Cracker: Architecture du programme}\\[0.4cm] % Title of your document
	
	\HRule\\[1.5cm]
	%\vspace{1.5cm}
	
	%------------------------------------------------
	%	Author(s)
	%------------------------------------------------
	
	%{\large\textit{Auteurs}}\\
	Groupe n\degree$105$\\[0.2cm]
	Amadéo \textsc{David}  - $44761700$% Your name
	
	Eduardo \textsc{Vannini} - $10301700$% Your name
	
	%------------------------------------------------
	%	Date
	%------------------------------------------------
	
	\vfill\vfill\vfill % Position the date 3/4 down the remaining page
	
	{\large 1\ier{} avril 2019} % Date, change the \today to a set date if you want to be precise
	
	%----------------------------------------------------------------------------------------
	
	\vfill % Push the date up 1/4 of the remaining page
	
\end{titlepage}


\section{Structures de données}
\label{sec:data_structures}

	\noindent
	Notre rapport comportera différentes structures de données, à savoir :
	
	\begin{itemize}
		\item 
		Les \textbf{mots de passe non-traités (hashes)}.
		Ceux-ci seront représentés par un tableau de trente-deux entiers sous la forme \lstinline{uint8_t} car un hash est encodé sur trente-deux bytes.
		Plus simplement, l'énoncé indique aussi que l'argument demandé par la fonction \lstinline{reversehash} doit être de la forme citée précédemment;
		
		\item 
		L'\textbf{ensemble des mots de passe non-traités}. Ceux-ci seront stockés dans un buffer (un tableau de pointeurs vers des tableaux de trente-deux \lstinline{uint8_t}) de taille prédéfinie. Il est à noter que si la taille du fichier à lire excède la capacité du buffer, la partie qui n'a pas pu être insérée sera ajoutée au buffer par la suite. Notons aussi que le buffer sera une structure de type \textit{Last In First Out};
		
		\item % Dans le .txt : liste chainée de "char*" - en réalité, plutôt des pointeurs vers des tableaux me semble-t-il?
		L'\textbf{ensemble des mots de passe constituants les \textit{meilleurs candidats} à un moment donné}.
		Ceux-ci seront stockés en clair dans une liste chaînée dont les données seront des pointeurs vers des tableaux de caractères (autrement dit des pointeurs de pointeurs vers des \lstinline{char}) représentant les versions déchiffrées des mots de passe retenus comme \textit{meilleurs candidats}.
		
		%\item %,note supplémentaire, peut-être pas utile?
		% En effet, c'est bien pour répondre à des questions éventuelles mais pas forcément utile dans le rapport.
		%\textbf{Note} : il est intéressant de remarquer qu'aucune structure n'est utilisée pour représenter un mot de passe car il n'est jamais utile d'accéder en même temps à la version codée et décodée d'un mot de passe en même temps.
		%Dès que celui-ci a été traité, sa version codée peut être oubliée.
		%Les mots de passe seront donc soit représentés par des \textit{int8\_t password[32]} comme cité précédemment, soit par des \textit{char*} de taille inconnue avant exécution du programme (les versions en clair).
		
	\end{itemize}
	
\section{Types de threads entrant en jeu}
	
	\noindent
	Notre programme comportera plusieurs types de threads, à savoir :
	
	\begin{itemize}
		\item
		Le \textbf{thread principal}.
		Il s'agit du thread dans lequel s'exécutera la fonction \lstinline{main} et qui permettra de gérer tous les autres threads du programme.
		Il sera donc le parent des threads décrits ci-dessous.
		Son rôle sera de récupérer les arguments dans \lstinline{argv} afin d'y réagir de manière appropriée, de créer les diverses ressources partagées ainsi que les mécanismes de protection de celles-ci (mutex, sémaphores, etc), de lancer et de terminer les threads de lecture et de calcul décrits plus bas.
		Avant que le programme ne termine son exécution, ce thread permettra aussi de gérer le "retour vers l'utilisateur", c'est-à-dire qu'il se chargera d'informer l'utilisateur du résultat des calculs;
		
		\item
		Les \textbf{threads de lecture}.
		Il s'agit de threads \textbf{producteurs} qui se chargeront de lire le fichier de hashes dans un buffer de taille fixe à déterminer.
		Il en existera un nombre prédéfini que nous devrons également déterminer à l'aide d'une analyse du remplissage du buffer lors de l'exécution.
		Ces threads devront être bloqués quand le buffer sera rempli et ils se termineront quand le fichier aura été lu dans son entièreté;
		
		\item
		Les \textbf{threads de calcul}.
		Il s'agit de threads \textbf{consommateurs} dont il existera $N$ instances, avec $N$ une valeur précisée en argument par l'utilisateur et dont la valeur par défaut sera $1$.
		Ces threads auront pour rôle d'inverser les hashes présents dans le buffer mentionné ci-dessus, puis de déterminer le "score" du mot de passe obtenu afin de pouvoir mettre à jour la liste des candidats en conséquence.
		Quand le buffer sera vide, ces threads seront bloqués mais ils ne se termineront pas tant que le fichier n'aura pas été lu dans son entièreté.
	\end{itemize}
	
	\noindent
	Les fonctions exécutées par nos threads auront un retour de type \lstinline{int} afin de pouvoir retourner des codes d'erreur en cas de besoin.

\section{Méthodes de communication entre les threads}
%\noindent
	Le programme utilisant le modèle des \textit{producteurs et consommateurs}, il faudra utiliser deux sémaphores.
	Le premier, appelé \lstinline{empty}, représentera le nombre de places libres dans le buffer et sera initialisé à une valeur \textit{M} égale à la taille du buffer.
	Le second sémaphore sera appelé \lstinline{full}, représentera le nombre de places occupées dans le buffer et sera initialisé à zéro.\\	
	
	Aussi, les différentes ressources partagées entre les threads et mentionnées dans la section \textit{\nameref{sec:shared_data}} seront protégées par des mutex.
 
\section{Informations communiquées entre les threads}
\label{sec:shared_data}
\noindent
	Au travers de variables partagées situées dans le heap, les threads se communiqueront les informations suivantes:
	
	\begin{itemize}
		\item
		L'\textbf{état de lecture du fichier}.
		Cette information sera communiquée entre les threads de lecture et de calcul.
		Elle sera décrite par une variable de type \lstinline{int} dont le nom sera "\lstinline{file_read}" et dont la valeur sera initialisée à \lstinline{0} puis mise à \lstinline{1} dès que le fichier aura été lu en entier.
		Cette modification de valeur se fera par le thread de lecture ayant détecté la fin du fichier;
		
		\item
		L'\textbf{indice du dernier hash} inséré dans le buffer.
		Cette information sera communiquée entre les threads de lecture et de calcul.
		Elle sera décrite par une variable de type \lstinline{int} et nommée "\lstinline{last_hash_index}" qui sera initialisée à la valeur \lstinline{-1} pour indiquer qu'aucun hash n'a encore été ajouté au buffer.
		Chaque ajout d'un hash au buffer se fera à la position \lstinline{last_hash_index + 1} (si celle-ci ne sort pas du buffer) et s'accompagnera de l'incrémentation de \lstinline{last_hash_index}, tandis que chaque inversion de hash se fera sur celui situé à la position \lstinline{last_hash_index} (si cette position est positive) et s'accompagnera d'une décrémentation de \lstinline{last_hash_index};
		
		\item 
		Les \textbf{hashes du buffer} constituent également des informations que se transmettront entre eux les threads de lecture et de calcul. Nous vous invitons à lire la section \textit{\nameref{sec:data_structures}} dans laquelle nous décrivons la structure du buffer;
		
		\item
		Les \textbf{meilleurs candidats} à un instant donné, contenus dans la liste chaînée décrite dans la section \textit{\nameref{sec:data_structures}}, ainsi que le \textbf{meilleur "score"} en ce même instant donné, représenté par une variable "\lstinline{best_score}" de type \lstinline{int}, constituent une information échangée par les threads de calcul entre eux, ainsi qu'avec le thread principal.
		
	\end{itemize}


\end{document}
